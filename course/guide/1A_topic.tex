%
% 1.1
%
% \section{Topics}%
%   %
%   %
%   \label{sec:topic}%
	\index{Methods}%
	%
	%

% 1.1.1
%
% \subsection{Quantitative methods}
%   %
%   %
	\label{sec:quantitative-methods}
  \index{Quantitative methods}%
  
	\newthought{Quantitative methods} designate a branch of social science methodology that applies statistical procedures to experimental or observational data in order to produce explanatory models for the complex, recurrent phenomena that affect populations and organizations. These methods can be used to analyze things like attitudes towards highly skilled and low-skilled immigration in the United States,\footcite{HainmuellerHiscox:2010a} the relationship between inequality and growth,\footcite{BanerjeeDuflo:2003} or the historical vote that put Adolf Hitler into power in interwar Germany.\footcite{KingRosen:2008a}%
		%
		%

    % why do we model (1): retrospective verification that the data does not contradict the theory so far
    % note: it's not forecasting because it's in-sample prediction
    % http://brenocon.com/blog/2013/08/tweet-shares-and-predicting-election-outcomes/

	The models produced by quantitative research account for the regularities that exist in the data by estimating how a set of independent, explanatory variables can predict the value of a dependent, outcome variable. To what extent, for example, is the prevalence of HIV/AIDS predictable from the level of economic inequality and degree of political unrest in a country? What is the impact of free textbooks on educational attainment? Does the support for violent action vary with age and education, in what direction, by how much, within what range and at what rate? A quantitative model can estimate these relationships, on top of which researchers develop theories to explain what causal paths are followed in the model.%
		%
		%

    % why do we model (2): state-driven (demography, public health, macroeconomics), market-based (finance, sports, polling)
    %
	A strong background assumption behind such questions is comparability. In quantitative data, the units of observation, such as individuals or countries, are defined through a set of commensurable characteristics—the variables. The first and perhaps most important requirement of a quantitative model is that you are measuring roughly the same thing among roughly similar units with sufficient reliability. This is far from obvious when you are aggregating, for instance, development statistics, because many countries have very low statistical capacity (among many other issues).\footcite{Jerven:2013a}%
		%
		%

	Each variable of a quantitative model is attached to a concept, like `household income' or `democratic status,' each of which provides an explanatory component to the research design. Measuring the effects of the `independent' variables on the `dependent' one is statistical jargon that means measuring the respective influence of each explanatory component on a given phenomenon, which might be a measure, like the number of children in a household or the GDP growth rate of a country, or the probability of occurence of an event, like abortion or state collapse.%
		%
		%

	\newthought{From a learning perspective,} what you can immediately figure out of the short description above is that quantitative methods require some attention to terminology: `data' are `observations' described by `variables' made of `values', some of which we `predict' from the others through the `estimation' of their `independent' effects. The topic also requires some (really light) exposure to logic and mathematics. Finally, any quantitative analysis requires some knowledge of the practical aspects of empirical research design, such as data collection or sampling.%
		%
		%

	Contrary to what bookshelves of statistics textbooks and horror stories about equations feeding on human flesh might have led you to believe, your learning approach of quantitative analysis should actually have more to do with practice than with theory. While you can rely on external knowledge to implement the theory of statistical models into your analysis, the practice of quantitative analysis implies a lot of iterative tinkering at the level of data preparation. At the technical level, you will have to draft and test things out through independent learning by trial-and-error.%
		%
		%
	Another practical dimension of quantitative analysis in the social sciences has to do with measurement and with the limited degree of precision of any social statistic, which makes issues of statistical significance secondary to issues like validity and reliability when it comes to social data. When your unit of analysis is a social one, start thinking in rounded figures: the measures are never more precise than what they are, nor the data more representative than what it can possibly be.%
		%
		\footnote{For that matter, `1.474\% of the general population' is almost never a credible statement with social data, not because a sample can never perfectly match the general population (only a large target subset of it), but rather because three-digit precision would indicate spectacularly precise estimation.}%
		%
		%


% 1.1.2
%
% \subsection{About this guide}%
%   %
%   %
	\newthought{This `Stata Guide'} was written as an introduction to Stata for students with a background in the social sciences. You are not expected to know anything about statistics, but you are expected to know a few things about social science research from your undergraduate curriculum. You should be familiar, for instance, with notions like cultural capital, gross domestic product per capita and political regimes.%
		%
		%

	We will cover the following topics:%

	\begin{itemize}
		\item This introduction deals with the course basics, essential computer skills and Stata fundamentals. It also explains how to set up a computer for the course.%
	
		\item %
		Section~\ref{ch:data} explains how to prepare data for analysis, and %
		Section~\ref{ch:distr} explains how to visualize distributions. This segment is primarily about description and univariate statistics. %
    It ends on instructions to submit your first draft.
	
		\item %
		Section~\ref{ch:asso} introduces statistical significance tests, and %
		Section~\ref{ch:ols} introduces ordinary least squares and simple linear regression. This segment is about association and bivariate statistics. %
    It ends on instructions to submit your revised draft.

		\item %
		Section~\ref{ch:lin} introduces multiple linear regression modelling, and %
		Section~\ref{ch:log} introduces logistic regression modelling. This segment covers the basics of statistical modelling for continuous and categorical variables in cross-sectional data. %

    The guide also includes an index including all commands cited in the text, and a list of bibliographic references.%
	\end{itemize}

\newthought{Several sections of the guide are still in draft form}, so watch for updates and read it along other documentation. Its writing started with students questions, and several sections were first written as short tutorials concerning specific issues with data management. One thing led to another, and we ended up with the current document. The aim is to cover 90\% of the course by version 1.0.%

	Several students have already provided very valuable feedback on the text—thanks, and keep the feedback flowing in!%

