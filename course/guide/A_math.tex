%
% A3
%
\chapter[Mathematics]{Mathematics for social scientists}%
  %
  %
  \label{ch:math}%
	\index{Mathematics}%
	%
	%




Many aspects of human survival are related to events, objects or people that can be enumerated. For example, infant mortality, forced migration and market collapse are all quantifiable social problems for which we might want to know how frequently they occur, and under what circumstances. A lot of applied mathematics is therefore used in the formalization of markets, populations and disease, both for industrial and for research purposes.

To manipulate things in mathematical form, we rely on symbols and conventions to write down expressions like $y = f(x)$ (the function $f$ that returns the value $y$ for a value of $x$).

%
\paragraph{Greek letters}%
%
Table~\ref{tbl:greek} references the Greek letters that are used in conventional notation: if you read a lot of science-fiction, you might be able to skip this, otherwise start familiarizing yourself with them. In practice, we will use only a subset of the Greek alphabet, and we will use only a few letters at a time.

\begin{table}%

$$
\begin{array}{r*5lrl}
\multicolumn{6}{l}{\text{\textbf{Lowercase}:}} &\multicolumn{2}{l}{\text{\textbf{Uppercase}:}}\\[.5em]
\alpha		& \text{alpha}			& \kappa	& \text{kappa}			& \tau		& \text{tau}		& \Gamma		& \text{Gamma} \\
\beta		& \text{beta}			& \lambda	& \text{lambda}			& \upsilon	& \text{upsilon}	& \Delta		& \text{Delta} \\
\gamma		& \text{gamma}			& \mu		& \text{mu}				& \phi		& \text{phi}		& \Theta		& \text{Theta} \\
\delta		& \text{delta}			& \nu		& \text{nu}				& \varphi	& \text{var-phi}	& \Lambda		& \text{Lambda} \\
\epsilon	& \text{epsilon}  		& \xi		& \text{xi}				& \chi		& \text{chi}		& \Xi			& \text{Xi} \\
\varepsilon	& \text{var-epsilon}	& \pi		& \text{pi}				& \psi		& \text{psi} 		& \Pi			& \text{Pi} \\
\zeta		& \text{zeta}			& \varpi	& \text{var-pi}			& \omega	& \text{omega}		& \Sigma		& \text{Sigma} \\
\eta		& \text{eta}			& \rho		& \text{rho}			&			&					& \Upsilon		& \text{Upsilon} \\
\theta		& \text{theta}			& \varrho	& \text{var-rho}		&			&					& \Phi			& \text{Phi} \\
\vartheta	& \text{var-theta}		& \sigma	& \text{sigma}			&			&					& \Psi			& \text{Psi} \\
\iota		& \text{iota} 			& \varsigma	& \text{var-sigma}		&			&					& \Omega		& \text{Omega}
\end{array}
$$
  
  \caption{Greek letters}\label{tbl:greek}
\end{table}

% You can read sequences of letters and numbers by simple enunciation: $\Delta x$ is pronounced ``delta-x'' and $\beta_1X_1$ is pronounced ``Beta-One X-One''. Note that the pronunciation of Greek letters varies from your linguistic standard: $\chi^2$, for example, designates the ``Chi-squared'' value, which is pronounced ``Kai-squared''.

%
\paragraph{Mathematical symbols}%
%
Table~\ref{tbl:math} references some common math symbols that are used to compose expressions such as: for every ($\forall$) nonnegative real number $(x \in \mathbb{R}^{+})$, there exists ($\exists$) a square root number given by the function $f: x \rightarrow \sqrt{x}$. Again, we will use only a few of these symbols for demonstration purposes.

\begin{table}%

$$
\begin{array}{rlrlrlrl}
\exists  	    & \text{there exists}	            & |x|      	  & \text{absolute value, determinant}			& \Sigma      & \text{summation}              \\
\forall		    & \text{for all/any}	            & [...]       & \text{closed interval}			            & \Pi         &  \text{product}               \\
\therefore		& \text{therefore}	              & (...)       & \text{open interval}    				        & \Delta x    & \text{the change in } x       \\
\Rightarrow		& \text{implies}			            & \{...\}     & \text{set}	                  			    & \partial	  & \text{partial differential}   \\
\to  & \text{goes to, approaches}  	            & \varnothing & \text{empty set}      		              & \int		    & \text{integral}               \\
\mid          & \text{such that, given}     	  & \subseteq   & \text{is a subset of}       			      & x!		      & \text{factorial}              \\
\text{iff}    & \text{if and only if}			      & \subset     & \text{is a proper subset of}			      & \ln(x)	    & \text{natural logarithm}      \\
\in           & \text{is an element of}		      & \cup        &	\text{union}                            & \log_b(x)		& \text{logarithm of base } b   \\
\notin        & \text{is not an element of} 		& \cap		    &	\text{intersection}                     &	x^k   	    & x \text{ at power/exponent } k\\
\approx       & \text{equals approximately}     & \infty      & \text{infinity}                         & \sqrt{x}    & \text{square root of } x      \\
\simeq        & \text{approximately equal to}   & \lim        & \text{limit}                            & \sqrt[3]{x} & \text{cubic root of } x       \\
\end{array}
$$
  
  \caption{Selected mathematical symbols and expressions}\label{tbl:math}
\end{table}


Some letters have standard representations in statistics, such as $\mu$ and $\bar X$ for the population and sample means, $\sigma$ and $s$ for standard deviation, $\sigma^2$ and $s^2$ for variance, etc. You will learn these conventions through practice. Using a computer will introduce slight deviations to these conventions because statistical software like Stata might denote, for instance, $\chi^2$ as \texttt{chi(2)}.

%
\paragraph{Numbers}%
%
Positive numbers like $6, 2, 1$ (the $+$ sign can be omitted) and negative numbers like $-4, -2$ are called integers. If a number $x$ has no floating point (decimals), you are looking at an integer.

The set of \emph{natural} numbers $\mathbb{N}$ contains positive integers $1,2,3,4, \ldots, $ and eventually contains $0$ if you need it to. These numbers can be ordered to form a coordinate system centred around $0$, as shown below:

\begin{figure}[h]
  \begin{tikzpicture}[x=0.75cm,>=stealth]
    \draw[<->] (-5,0)--(5,0);
    \foreach \x in {-4,...,4}
    \draw[shift={(\x,0)},color=black] (0pt,2pt) -- (0pt,-2pt) node[below] {\footnotesize $\x$};
    \node[above,color=s1] at (7.5,15pt)  {irrational numbers};
    \draw[color=s1] (1.41,0) -- (1.41,15pt);
    \node[above,color=s1] at (1.41,15pt) {$\sqrt{2}$};
    \draw[color=s1] (2.71,0) -- (2.71,15pt);
    \node[above,color=s1] at (2.71,15pt) {$e$};
    \node[below,color=s2] at (7.5,-15pt)  {rational numbers};
    \draw[color=s2] (-1/2,0) -- (-1/2,-15pt);
    \node[below,color=s2] at (-1/2,-15pt) {$-\frac{1}{2}$};
    \draw[color=s2] (9/4,0) -- (9/4,-15pt);
    \node[below,color=s2] at (9/4,-15pt) {$\frac{9}{4}$};
    \node[below] at (-5,-5pt)  {$\ldots$};
    \node[below] at (5,-5pt)   {$\ldots$};
  \end{tikzpicture}
  \caption{The real number line.}
\end{figure}

The numbers covered by that coordinate system are said to be rational if they can be expressed as a ratio of two integers $r = \frac{n}{d}$ with $d \neq 0$. Numbers that cannot take the form of a ratio, like $\sqrt{2}$ or the mathematical constant $e$, are said to be irrational.

Numbers belong to sets. The set of \emph{real} numbers $\mathbb{R}$ contains the continuum of all rational and irrational numbers. This set designates a system of coordinates called the real number line, from which you get intervals, Cartesian planes, and the rest of the mathematics that we will use.

%
%
%
\section{Functions}

\newthought{A function} is a rule that describes a relationship between numbers. For each number x, a function assigns a unique number y according to some rule. Thus a function can be indicated by describing the rule, as “take a number and square it,” or “take a number and multiply it by 2,” and so on. We write these particular functions as y = x2, y = 2x. Functions are sometimes referred to as transformations.%

%
%
\subsection{Operators}

%
\paragraph{Summation}%
  %
  When you add all values of $X$ over $n$ observations $X_1, X_2, \ldots, X_n$, you are \emph{aggregating} the values of $X$ over $i = 1, 2, \ldots, n$ observations. Summation notation uses the $\Sigma$ letter to represent this operation:%

  $$\sum_{i=1}^n X_i = X_1 + X_2 + \cdots + X_n$$

  The $X_1 + X_2$ relationship is \emph{additive}, and a subtraction is simply a negative addition: $a-b=a+(-b)$. The identity number for addition and subtraction is 0: if you add or subtract 0 to a value, nothing happens to that value (it stays equal).%

\paragraph{Product}%
  %
  When you multiply all values of $X$ over $n$ observations $X_1, X_2, \ldots, X_n$, you are looking at the \emph{product} of $X$ over $i = 1, 2, \ldots, n$ observations. Product notation uses the $\Pi$ letter to represent this operation:%

  $$\prod_{i=1}^n X_i = X_1 \times X_2 \times \cdots \times X_n$$

  The $X_1 \times X_2$ relationship is \emph{multiplicative}, and a division is simply an inverse multiplication: $\frac{a}{b} = a \cdot \frac{1}{b}$. Their identity number is 1: if you multiply or divide a value by 1, nothing happens to that value.%

  % If the world were organized as in the movie \emph{Highlander}, the ultimate issue is the survival of one, hence the memorable tagline of the movie: ``There can be only one''. The limit of the \emph{Highlander} universe is 1 because every fight between two Immortals systematically leads to one less living Immortal. The limit of $x/2$ for $x > 1$ is 1.

%
%
\subsection{Exponentials}

Power functions feature an exponent $k$ to raise $x$ to a given power, such that $f(x) = x^k = x*x*...*x$ ($k$ times). Some examples of power functions, where the exponent is constant, are shown in Figure~\ref{fig:exps-and-powers}.%

%
\paragraph{Example: Scientific notation}%
%
Power functions are useful to rewrite large numbers with the function $f: x \to a \times 10^b$, where $a$ is any real number and $b$ is an integer. For example, by the end of August 2012, the U.S. national debt amounted to \$\href{http://www.treasurydirect.gov/NP/BPDLogin?application=np}{16,015,769,788,215.80}. This number can also be written as $16 \times 10^{12}$, \ie 16 trillion, or as \ensuremath{1.6\times 10^{13}} in scientific notation.

%
\paragraph{Exponentials}

Exponential functions are also useful to handle some phenomena like the exponential growth of the population or labour force in many countries, or the exponential decay of an organism or technology.





