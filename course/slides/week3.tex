% Definitions
% Issues
% Recoding
% Practice
  % Example: WVS dataset
  % Practice session
  % Exercises

\documentclass[t]{beamer}
\usetheme{hkllite}

\title{variables}
	\author{François Briatte \& Ivaylo Petev}
	\date{Week~\#3}

\begin{document}

  \frame[plain]{
		\titlepage\\[7.25em]
		\begin{columns}[T]
			\column{.50\textwidth}
				% \tableofcontents[hideallsubsections]
			\column{.35\textwidth}
				\includegraphics[width=\textwidth]{pie-no}
		\end{columns}
	}
	%
	%

	\begin{frame}[t]{Coursework: prepare for the \red{first draft}}

    \begin{block}{Roadmap}
      \begin{enumerate}
        \item Catch up on any \textbf{late work}
        % Spring 2019 link
        % \item \red{\href{https://frama.link/srqm-2019}{Register your \textbf{group}}} on Google Sheets%
        \item Manage your \textbf{group work}
        \item Provisional \textbf{deadline: \red{Wednesday 2 March}}
      \end{enumerate}
    \end{block}

    \begin{block}{Instructions and help}
      \begin{itemize}
	      \item \textbf{Course emails}, Sessions 1--4
        \item \textbf{Stata~Guide}, Sections 5--9%
        \item First draft \textbf{templates} (to be sent by email)
        \item \textbf{Q \& A} in Sessions 4 and 5
      \end{itemize}
    \end{block}
    		
	\end{frame}
  %
  %

  % %
  % %
  % \section{Definitions}
  % %
  % %
  %
  % \begin{frame}[t]{Definitions}
  %
  %      \begin{block}{Mathematical foundations}
  %
  %         \begin{itemize}
  %           \item \textbf{Random variables:} numbers assigned to states
  %           \item \textbf{Continuous variables:} ranges of values
  %           \item \textbf{Discrete variables:} sets of values
  %         \end{itemize}
  %
  %      \end{block}
  %
  %    \begin{block}{Dimensions and types}
  %
  %      \begin{itemize}
  %        \item \textbf{Continuous:} infinite/large number of possible values
  %        \item \textbf{Categorical:} finite/small number of possible values
  %        \item \textbf{Ordinal:} meaningful ordering of the values
  %      \end{itemize}
  %
  %    \end{block}
  %
  % \end{frame}
  %   %
  %   %
	%
	% %
	% %
	% \section{Issues}
	% %
	% %
	%
	% \begin{frame}[t]{Issues}
	%
	%    \begin{block}{Measurement}
	%
	%      \begin{itemize}
	% 			 \item \textbf{Availability} of the data
	% 	     \item \textbf{Accuracy} of the measurement
	% 	     \item \textbf{Meaningfulness} of the unit
	%      \end{itemize}
	%
	%      % Examples: unit-less indexes, baseline growth rates, `standardized' Likert scales, breaks in methodology and/or availability, …
	%
	%    \end{block}
	%
	%      \begin{block}{Coding}
	%
	%         \begin{itemize}
	%           \item \textbf{Values:} \texttt{-1 0.7 9} (numeric), \texttt{"UK"} (string), \texttt{.} (missing)
	%       		\item \textbf{Variable labels:} \texttt{"Fertility rate"}
	%       		\item \textbf{Value labels:} \texttt{1 "White" 2 "Black" 3 "Hispanic" ...}
	%         \end{itemize}
	%
	%      \end{block}
	%
	% \end{frame}
	% %
	% %

  % %
  % %
  % \section{Recoding}
  %   %
  %   %
  %
  % \begin{frame}[t]{Recoding}
  %
  %     \begin{block}{Rationale}
  %
  %   \begin{itemize}
  %     \item \textbf{Create groups}, e.g. age cohorts, social classes
  %     \item \textbf{Create dummies}, i.e. binary (0/1), true/false indicators
  %     \item \textbf{Change encoding} for missing values
  %   \end{itemize}
  %
  %     \end{block}
  %
  %     \begin{block}{Stata commands}
  %
  %     \begin{itemize}
  %       \item Create variables: \code{gen}, \code{tab, gen()}, \code{clonevar}, …
  %       \item Recode values: \code{recode}, \code{irecode}, \code{replace}, …
  %       \item Assign labels: \code{la var}, \code{la def}, \code{la val}, …
  %     \end{itemize}
  %
  %     \end{block}
  % \end{frame}
  % %
  % %

	% %
	% %
	% \section{Practice}
	% %
	% %
		
	% \subsection{Example: WVS dataset}
	
	\begin{frame}[t]{Practice: \red{WVS 2000 dataset}}

		\begin{quote}
		``[The government] should implement only \\
		the laws of the sharia.''\\[1em]
		\end{quote}
		
		\begin{itemize}
			\item Measured on a 5-point scale of agreement
			\item Asked in 9 African and Asian countries between 1999 and 2004
		\end{itemize}

		\vspace{1em}
		
    Data:
	
			\begin{columns}[c]
				\column{.725\textwidth}
				
				\begin{itemize}
					\item \href{http://www.worldvaluessurvey.org/wvs.jsp}{World Values Survey} (WVS)%
					\item Sample: resident populations aged 15+
				\end{itemize}
	
				\column{.2\textwidth}
				\href{http://www.worldvaluessurvey.org/wvs.jsp}{\includegraphics[width=\textwidth]{logo-wvs}}%
			\end{columns}
	
	\end{frame}
	%
	%

  % % %
  % % %
  % % \subsection{Coursework}
  % %   %
  % %   %
  %
  %  \begin{frame}[t]{Coursework \red{for next week}}
  %
  %     \begin{block}{Replication}
  %     Open the do-file that we started in class:
  %     \begin{center}
  %       \code{doedit code/week3}
  %     \end{center}
  %
  %       Replicate the full analysis by \textbf{executing all commands} in the do-file, \textbf{reading all comments} as you go along.%
  %     \end{block}
  %
  %     \begin{alertblock}{Coursework}
  %       \begin{itemize}
  %        \item \textbf{Read the Stata Guide}, Sections 5--8.
  %        \item \textbf{Meet with your group} to start discussing your project.
  %        \item \textbf{Start writing code} to describe and prepare variables.
  %       \end{itemize}
  %     \end{alertblock}
  %
  % \end{frame}
  %   %
  %   %

  \subsection{Learning goals}
  %
  %

 \begin{frame}[t]{Learning goals for this week}

   \code{doedit code/week3}\\[1em]

	\begin{itemize}
		\item Subsetting to a \red{cross-section} of \red{complete observations}\\[.5em]%
      Ex. commands: \code{drop} · \code{if mi()} · \code{misstable pat}\\[.5em]%
		\item Creating and \red{summarizing} variables\\[.5em]%
      Ex. commands: \code{fre} (better than \code{tab}) · \code{su} · \code{tabstat}\\[.5em]%
		\item \red{Recoding} variables of different \red{variable types}\\[.5em]%
      Ex. commands: \code{clonevar} · \code{replace} · \code{recode, gen()}\\[.5em]%
	\end{itemize}

    \begin{block}{Side note}
       Graphs are useful, but Stata graph syntax is awkward.
    \end{block}

  \end{frame}
  %
  %
  
\end{document}