% Correlation
% Correlation matrixes
% Practice
  % Example: QOG dataset
  % Practice session
  % Exercise

\documentclass[t]{beamer}
\usetheme{hkllite}

\title{correlation}
	\author{François Briatte \& Ivaylo Petev}
	\date{Week~\#7}

\begin{document}
	
  \frame[plain]{
		\titlepage\\[7.25em]
		\begin{columns}[T]
			\column{.4\textwidth}
				\tableofcontents[hideallsubsections]
			\column{.5\textwidth}
				\hfill %
				\href{http://replicatedtypo.com/wp-content/uploads/2012/11/ChocolateSerialKillers_WintersRoberts.pdf}%
					{\includegraphics[width=\textwidth]{correlation-chocolate-murders}}		
		\end{columns}
	}
	%
	%
	
	\section{Correlation}

	\begin{frame}[c]

		\begin{center}
			\includegraphics[width=\textwidth]{hamster-contraception}	
		\end{center}
		
		\vfill \par Source: Harkness, ``\href{http://onlinelibrary.wiley.com/doi/10.1111/j.1740-9713.2012.00549.x/abstract}{Seduced by Stats?}'', \emph{Significance}, 2012.
		
	\end{frame}

	%
	%
  
  \begin{frame}[c]

		\begin{center}
			\includegraphics[height=6.5cm]{chocolate-nobel}
		\end{center}
		
		\vfill \par Source: Messerli, ``\href{http://www.nejm.org/doi/pdf/10.1056/NEJMon1211064}{Chocolate Consumption, Cognitive Function, and Nobel Laureates}'', \emph{New England Journal of Medicine}, 2012.
		
	\end{frame}

	%
	%
		
    \begin{frame}[c]{Linear correlation} %%% \thesection.~
	
		\begin{center}
			\includegraphics[height=5.5cm]{correlation-scatterplot}
		\end{center}
    
    Measured numerically using (Pearson's) linear correlation coefficient, $\rho$ (based on variance and covariance).
				
	\end{frame}
	
  % \begin{frame}{Pearson correlation coefficient}
  %
  %   \begin{block}{Measuring association as the linear dependence of two variables:}
  %
  %     \begin{align*}
  %       \text{Population notation} \quad
  %       \rho &= \frac{\text{Cov}(X,Y)}{\text{Var}_X\text{Var}_Y}, \quad
  %       -1 \leq \rho \leq 1
  %       \\
  %       \text{Sample notation} \quad
  %       r &= \frac{1}{n-1} \sum ^n _{i=1} (\frac{X_i - \bar{X}}{s_X}) (\frac{Y_i - \bar{Y}}{s_Y})
  %     \end{align*}
  %
  %   \end{block}
  %
  %   \begin{alertblock}{Detects linear correlation}
  %
  %     \begin{itemize}
  %       \item Uncorrelated $\neq$ unrelated
  %       \item Correlated $\neq$ unconfounded
  %     \end{itemize}
  %
  %   \end{alertblock}
  %
  % \end{frame}
  % %
  % %

	\begin{frame}[c]{Perfect (positive, negative) correlation}
			
		\begin{center}
			\includegraphics[width=\textwidth]{correlation-perfect}		
		\end{center}

	\end{frame}	

	\begin{frame}[c]{Significant (moderate, strong) correlation}
			
		\begin{center}
			\includegraphics[width=\textwidth]{correlation-significant}		
		\end{center}

	\end{frame}	

	\begin{frame}[c]{Insignificant (weak, non-linear) correlation}
			
		\begin{center}
			\includegraphics[width=\textwidth]{correlation-insignificant}		
		\end{center}

	\end{frame}
	%
	%
	
  % \begin{frame}{Pearson correlation coefficient}
  %
  %   \begin{block}{Significance test:}
  %
  %     \begin{align*}
  %       \text{Null hypothesis~} H_0 \quad
  %       r &= 0
  %       \\
  %       \text{Test statistic} \quad
  %       T &= r \sqrt{\frac{n-2}{1-r^2}}
  %     \end{align*}
  %
  %   \end{block}
  %
  %   \begin{alertblock}{Sanity check}
  %
  %     \begin{itemize}
  %       \item Uncorrelated $\neq$ independent
  %       \item Correlated $\neq$ causally related
  %     \end{itemize}
  %
  %   \end{alertblock}
  %
  % \end{frame}
  % %
  % %
	
%   \section{Correlation matrixes}
%
%   \begin{frame}[c]{\thesection.~Correlation matrixes}
%
%     \begin{block}{\texttt{pwcorr [varlist], [obs sig]}}
%
%           \begin{itemize}
%             \item \texttt{obs} shows the number of observations
%         \item \texttt{sig} shows the coefficient's $p$-value
%           \end{itemize}
%
%     \end{block}
%
%     \begin{block}{\texttt{gr mat [varlist], [half etc.]}}
%
%           \begin{itemize}
%             \item \texttt{half} plots only half of all graphs (quicker)
%             \item accepts scatterplot options (\texttt{jitter}, \texttt{mlab}, etc.)
%           \end{itemize}
%
%     \end{block}
%
%     \begin{block}{\texttt{mkcorr [varlist], lab num sig log(file.txt) replace}}
%       \begin{itemize}
%         \item to install: \texttt{ssc install mkcorr}
%         \item to understand the options: \texttt{help mkcorr}
%       \end{itemize}
%     \end{block}
%
%   \end{frame}
%
%   %
%   %
%
%   %   \begin{frame}[c]{Correlation matrixes}
%   %
%   %   \begin{block}{\texttt{mkcorr [varlist], lab num sig log(file.txt) replace}}
%   %     \begin{itemize}
%   %       \item to install: \texttt{ssc install mkcorr}
%   %       \item to understand the options: \texttt{help mkcorr}
%   %     \end{itemize}
%   %   \end{block}
%   %
%   %   \begin{alertblock}{Computer skills}
%   %     \begin{itemize}
%   %       \item Import as a table in a spreadsheet editor.
%   %       \item Convert from text to table in a rich text editor.
%   %     \end{itemize}
%   %
%   %   \end{alertblock}
%   %
%   % \end{frame}
%   %
%   % %
%   % %
%
%   % \begin{frame}[c]{One variable, many predictors}
%   %
%   %   \begin{center}
%   %     \includegraphics[width=.7\textwidth]{hate-sc1}
%   %   \end{center}
%   %
%   %   \par Source: Florida, ``\href{http://www.theatlantic.com/national/archive/2011/05/the-geography-of-hate/238708/}{The Geography of Hate}'', \emph{The Atlantic}, 2011.
%   %
%   % \end{frame}
%   %
%   % \begin{frame}[c]{One variable, many predictors}
%   %
%   %   \begin{center}
%   %     \includegraphics[width=.7\textwidth]{hate-sc2}
%   %   \end{center}
%   %
%   %   \par Source: Florida, ``\href{http://www.theatlantic.com/national/archive/2011/05/the-geography-of-hate/238708/}{The Geography of Hate}'', \emph{The Atlantic}, 2011.
%   %
%   % \end{frame}
%   %
%   % \begin{frame}[c]{One variable, many predictors}
%   %
%   %   \begin{center}
%   %     \includegraphics[width=.7\textwidth]{hate-sc3}
%   %   \end{center}
%   %
%   %   \par Source: Florida, ``\href{http://www.theatlantic.com/national/archive/2011/05/the-geography-of-hate/238708/}{The Geography of Hate}'', \emph{The Atlantic}, 2011.
%   %
%   % \end{frame}
%   %
%   % \begin{frame}[c]{One variable, many predictors}
%   %
%   %   \begin{center}
%   %     \includegraphics[width=.7\textwidth]{hate-sc4}
%   %   \end{center}
%   %
%   %   \par Source: Florida, ``\href{http://www.theatlantic.com/national/archive/2011/05/the-geography-of-hate/238708/}{The Geography of Hate}'', \emph{The Atlantic}, 2011.
%   %
%   % \end{frame}
%   %
%   % \begin{frame}[c]{One variable, many predictors}
%   %
%   %   \begin{center}
%   %     \includegraphics[width=.7\textwidth]{hate-sc5}
%   %   \end{center}
%   %
%   %   \par Source: Florida, ``\href{http://www.theatlantic.com/national/archive/2011/05/the-geography-of-hate/238708/}{The Geography of Hate}'', \emph{The Atlantic}, 2011.
%   %
%   % \end{frame}
%
%   %
%   %
%
%   \begin{frame}[t]{Numerically, with \code{pwcorr}}
%
%     \begin{tikzpicture}
%       \node[anchor=south west,inner sep=0] (image) at (0,0) {\includegraphics[width=.8\textwidth]{pwcorr}};
%
%         \begin{scope}[x={(image.south east)},y={(image.north west)}]
%         \draw[fill=blue, opacity=0.25] (.32, .11) rectangle (.42, .17);
%         \draw[fill=red, opacity=0.25] (.43, .05) rectangle (.53, .11);
%         \draw[fill=green, opacity=0.25] (.54, -.01) rectangle (.64, .05);
%             \node[anchor=west, fill=blue, text opacity=1, opacity=0.25] at (.8, .22) { coefficient };
%             \node[anchor=west, fill=red, text opacity=1, opacity=0.25] at (.8, .10) { $p$-value };
%             \node[anchor=west, fill=green, text opacity=1, opacity=0.25] at (.8, -.02) { observations };
% %        \draw[help lines,xstep=.1,ystep=.1] (0,0) grid (1,1);
% %        \foreach \x in {0,1,...,9} { \node [anchor=north] at (\x/10,0) {0.\x}; }
% %        \foreach \y in {0,1,...,9} { \node [anchor=east] at (0,\y/10) {0.\y}; }
%         \draw[fill=yellow, opacity=0.25] (.19, .42) rectangle (.31, .58);
%             \node[anchor=west, fill=yellow, text opacity=1, align=left, opacity=0.25] at (.8, .58) { $r = -.2$ \\ $p < .02$ \\ $N = 140$ };
%         \end{scope}
%     \end{tikzpicture}
%
%   \end{frame}
%
%   \begin{frame}[t]{Visually, with \code{gr mat}}
%
%     \begin{center}
%       \includegraphics[width=\textwidth]{correlation-matrix}
%     \end{center}
%
%     \vfill \par Source: Adhikari \emph{et al.}, ``Public Policy, Political Connections, and Effective Tax Rates: Longitudinal Evidence from Malaysia'', \emph{Journal of Accounting and Public Policy}, 2006.
%
%   \end{frame}
%   %
%   %

	%
	%
	\section{Practice}
	%	
	%

	%
	%
	\subsection{Example: QOG dataset}
	%
	%
	
	\begin{frame}[t]{Practice: \red{QOG dataset}}

		\begin{columns}[c]
			\column{.55\textwidth}

	    Data:\\[.5em]

			\begin{itemize}
				\item Quality of Government (QOG)
				\item Sample: countries, c.~2012
			\end{itemize}
		
			\vspace{.75em}
		
	    Variables:\\[.5em]
		
			\begin{itemize}
				\item Fertility rate
				\item Education years
				\item Corruption Perceptions Index
				\item Human Development Index
				\item \% Female ministers
			\end{itemize}
	
			\column{.35\textwidth}

			\includegraphics[width=\textwidth]{logo-qog}

		\end{columns}
	
	\end{frame}
	%
	%

 % \begin{frame}[t]{Learning goals for this week}
 %
 %   \code{doedit code/week7}\\[1em]
 %
 %   \begin{itemize}
 %     \item \red{Correlating} two variables\\[.5em]%
 %      Ex. commands: \code{pwcorr}, \code{sc}, \code{gr mat}\\[.5em]%
 %     \item Looking at \red{bivariate} relationships in \red{crosstabulations}\\[.5em]%
 %      Ex. commands: \code{tab} and its many options\\[.5em]%
 %     \item Reading \red{Chi-squared tests}\\[.5em]%
 %      Ex. commands: \code{tab, chi2} and \code{tabchi}\\[.5em]%
 %   \end{itemize}
 %
 %    \begin{block}{Useful to interpret $t$-test results}
 %       \url{https://stats.idre.ucla.edu/stata/output/t-test/}
 %    \end{block}
 %
 %  \end{frame}
 %  %
 %  %	

  % %
  % %
  % \subsection{Practice session}
  %   %
  %   %
  %
  % \begin{frame}[t]{Practice session}
  %
  %     \begin{block}{Class}
  %       \comm{Get the do-file for this week.}\\
  %       \code{srqm\_get week7.do}\\
  %
  %     \comm{Open to read and replicate.}\\
  %     \code{doedit code/week7}\\
  %     \end{block}
  %
  %     \begin{alertblock}{Coursework}
  %       \begin{itemize}
  %       \item Finish the do-file and read all comments at home.
  %       \item Correct your do-file and add significance tests.
  %       \item Correct your paper and substantiate its hypotheses.
  %       \end{itemize}
  %     \end{alertblock}
  %
  % \end{frame}
  %   %
  %   %
		
\end{document}
