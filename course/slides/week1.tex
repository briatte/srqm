% The topic
  % Reality
  % Data
  % Interpretation
  % Bias
% The course
  % Essentials
  % Mechanics
  % Homework
  % Logistics
% Notes
  % On computers
  % On software
  % On slides
  % On emails
% Coursework

\documentclass[t]{beamer}
\usetheme{hkllite}

\title{introduction}
  \author{François Briatte \& Ivaylo Petev}
  \date{Week~\#1}

\begin{document}

  \frame[plain]{
    \titlepage\\[7.25em]
    \begin{columns}[T]
      \column{.50\textwidth}
        \tableofcontents[hideallsubsections]
      \column{.35\textwidth}
        \includegraphics[width=\textwidth]{stats-evitons}
    \end{columns}
  }
  %
  %

  %
  %
  \section{The topic}
  %
  %

  %
  %
  \subsection{Reality}
  %
  %

  \begin{frame}[t]{Reality is \red{predictable}}
    \href{http://articles.latimes.com/2010/aug/21/local/la-me-predictcrime-20100427-1}{\includegraphics[width=\textwidth]{predicting}}
  \end{frame}
  %
  %

  \begin{frame}[t]{Reality is \red{visualizable}}
    \href{http://www.bricoleurbanism.org/whimsicality/urban-fabric-form-comparison/}{\includegraphics[width=\textwidth]{visualize}}
  \end{frame}
  %
  %

  \begin{frame}[t]{Reality is \red{multidimensional}}
    \href{http://www.last.fm/user/phnk1/library}{\includegraphics[width=\textwidth]{lastfm}}
  \end{frame}
  %
  %

  % \begin{frame}[c]{Reality is \red{relational}}
  %
  %   \begin{columns}[T]
  %
  %     \column{.45\textwidth}
  %
  %     \begin{center}
  %       \includegraphics[height=4cm]{friendwheel}\\
  %       \vspace{0.74cm}
  %       Friendship ties on Facebook
  %     \end{center}
  %
  %     \column{.45\textwidth}
  %
  %     \begin{center}
  %       \href{http://www.sociology.columbia.edu/pdf-files/bearmanarticle.pdf}{\includegraphics[height=4cm]{network}}\\
  %       \vspace{0.7cm}
  %       Sexual ties in high school
  %     \end{center}
  %
  %   \end{columns}
  %
  % \end{frame}
  % %
  % %

  %
  %
  \subsection{Data}
  %
  %

  \begin{frame}[t]{\red{Data} are professional assets}
    \includegraphics[width=\textwidth]{oecd}
  \end{frame}
  %
  %

  % \begin{frame}[c]{Data stand as \red{policy expertise}}
  %
  %   \begin{center}
  %     \href{http://www.securite-sociale.fr/chiffres/ccss/notesconj/conj200903.pdf}{\includegraphics[width=.75\textwidth]{ondam}}
  %   \end{center}
  %
  % \end{frame}
  % %
  % %

  %
  %
  \subsection{Interpretation}
  %
  %

  % \begin{frame}[c]{\red{Interpretation} is key to all analysis}
  %
  %   \begin{columns}[T]
  %
  %     \column{.45\textwidth}
  %
  %     \begin{center}
  %       \href{http://jhfowler.ucsd.edu/alone_in_the_crowd.pdf}{\includegraphics[height=4cm]{fowler-network}}\vspace{1cm}
  %       Loneliness in social networks
  %     \end{center}
  %
  %     \column{.45\textwidth}
  %
  %     \begin{center}
  %       \href{http://books.google.fr/books?id=gvgCYyFN7RIC&pg=PA19&lpg=PA19}{\includegraphics[height=4cm]{jesus}}\vspace{1.05cm}
  %       Sets of Christian beliefs
  %     \end{center}
  %
  %   \end{columns}
  %
  % \end{frame}
  % %
  % %

  \begin{frame}[c]{\red{Interpretation} takes practice}

    \begin{columns}[T]

      \column{.45\textwidth}

      \begin{center}
        \href{http://jhfowler.ucsd.edu/alone_in_the_crowd.pdf}{\includegraphics[height=5cm]{fowler-social-distance}}\\
        \vspace{0.25cm}
        With explanation
      \end{center}

      \column{.45\textwidth}

      \begin{center}
        \includegraphics[height=5cm]{explain}\\
        \vspace{0.25cm}
        Without explanation
      \end{center}

    \end{columns}

  \end{frame}
  %
  %

  % \begin{frame}[t]{\red{Interpretation} is what this course is eventually about}
  %
  %   \begin{center}
  %     \href{http://www.phdcomics.com/comics.php?f=1219}{\includegraphics[width=.8\textwidth]{phdcomics-productivity}}
  %   \end{center}
  %
  %   \begin{exampleblock}{Questions}
  %
  %     \begin{itemize}
  %       \item What is the \textbf{measurement} method for each axis?
  %       \item What is the \textbf{probability} of 2am being the cutoff point?
  %       \item What is the \textbf{shape} of the time/productivity relationship?
  %     \end{itemize}
  %
  %   \end{exampleblock}
  %
  % \end{frame}
  % %
  % %

  %
  %
  \subsection{Bias}
  %
  %

  % \begin{frame}[t]{\red{Bias} and measurement}
  %
  %   \begin{block}{Observational data}
  %
  %     \begin{itemize}
  %       \item Survey design
  %       \item Sampling strategy
  %       \item Question wording
  %     \end{itemize}
  %
  %   \end{block}
  %
  %   \begin{block}{Official statistics}
  %
  %     \begin{itemize}
  %
  %       \item Unreliable aggregates       % e.g. GDP in China
  %       \item Low statistical capacity    % e.g. "Poor Numbers" in Africa
  %         \item Ecological fallacies        % lack of granularity
  %
  %     \end{itemize}
  %
  %   \end{block}
  %
  % \end{frame}
  % %
  % %
  %
  % \begin{frame}[t]{\red{Bias} and manipulation}
  %
  %   \begin{columns}[T]
  %
  %     \column{.3\textwidth}
  %
  %     \begin{itemize}
  %       \item Media coverage        % e.g. climate change
  %       \item Political spin        % e.g. unemployment and educational achievement in France
  %       \item Policy implications   % e.g. deficit figures and the IMF bailout in Greece
  %     \end{itemize}
  %
  %     Add to that:
  %
  %     \begin{itemize}
  %       \item \href{http://andrewgelman.com/2012/08/scientific-fraud-double-standards-and-institutions-protecting-themselves/}{scientific fraud},
  %       \item \href{http://andrewgelman.com/2012/10/ethical-standards-in-different-data-communities/}{data ethics}, % or lack thereof
  %       \item \href{https://en.wikipedia.org/wiki/Merchants_of_Doubt}{doubt-mongering}, % and the market for skepticism
  %       \item publishing bias, … % in journals and clinical trials
  %     \end{itemize}
  %
  %     \column{.6\textwidth}
  %
  %       \href{https://metofficenews.wordpress.com/2010/07/28/unmistakable-signs-of-a-warming-world/}%
  %       {\includegraphics[width=.9\textwidth]{warming}}
  %
  %   \end{columns}
  %
  % \end{frame}
  % %
  % %

  \begin{frame}[t]{\red{Bias} requires critical skills}

    \href{http://www2.psych.ubc.ca/~henrich/pdfs/WeirdPeople.pdf}{\includegraphics[width=\textwidth]{bbs.pdf}}

  \end{frame}
  %
  %

  \begin{frame}[c, plain]{}

    \absoluteimage{\href{https://en.wikipedia.org/wiki/The_Incredulity_of_Saint_Thomas_(Caravaggio)}{\includegraphics[width=\paperwidth]{caravaggio}}}

  \end{frame}
  %
  %

  %
  %
  \section{The course}
  %
  %

  %
  %
  \subsection{Essentials}
  %
  %

  %
  %
  \subsection{On software}

  \begin{frame}[t]{Course \red{essentials}}

    \begin{columns}[T]

      \column{.7\textwidth}

      We will use \textbf{\red{\href{http://www.stata.com/}{Stata}}} to learn basic\\[1em]%

      \begin{enumerate}
        \item \textbf{Data} management
        \item Statistical \textbf{estimation} and \textbf{inference}
        \item Regression \textbf{models}
      \end{enumerate}

      \column{.3\textwidth}

      \begin{center}
        \vspace{-2em}
        \href{http://www.stata.com/}{\includegraphics[width=.8\textwidth]{icon-stata16}}
      \end{center}

    \end{columns}

      \vspace{1em}

      \begin{block}{Core teaching blocks}
        \begin{itemize}
          \item Statistical theory      \hfill … textbook \textbf{readings}
          \item Statistical computing   \hfill … Stata do-files (\textbf{code})
          \item Lots of social science  \hfill … example \textbf{papers}
        \end{itemize}
      \end{block}

      % \begin{alertblock}{Requirements}
      %
      %   To operate Stata efficiently, you need
      %
      %   \begin{itemize}
      %     \item to use a fairly recent computer with a bit of disk space
      %       % ... system requirements
      %     \item to understand how files are organized on your hard drive
      %       % ... folder paths
      %     \item to type and `run' commands that follow a specific syntax
      %       % ... keyboard shortcuts
      %   \end{itemize}
      %
      % \end{alertblock}

  \end{frame}
  %
  %

  % \begin{frame}[t]{Course \red{essentials}}
  %
  %   \begin{block}{Core learning objectives}
  %     \begin{enumerate}
  %       \item Data management
  %       \item Statistical estimation
  %       \item Regression modelling
  %     \end{enumerate}
  %   \end{block}
  %
  %   \begin{block}{Core teaching blocks}
  %     \begin{itemize}
  %       \item Statistical theory      \hfill … textbook \textbf{readings}
  %       \item Statistical computing   \hfill … Stata do-files (\textbf{code})
  %       \item Tons of social science  \hfill … example \textbf{papers}
  %     \end{itemize}
  %   \end{block}
  %
  % \end{frame}
  % %
  % %

  % %
  % %
  % \subsection{Mechanics}
  % %
  % %

  % \begin{frame}[t]{Course \red{mechanics}}
  %
  % \begin{columns}[T]
  %
  %   \column{.6\textwidth}
  %
  %   \begin{block}{Requirements}
  %     \begin{itemize}
  %       \item Attendance
  %       \item Homework
  %       \item No plagiarism
  %     \end{itemize}
  %   \end{block}
  %
  %   \begin{alertblock}{Grading}
  %     \begin{itemize}
  %       \item Code and paper
  %       \item Draft, revised, final
  %       \item Project management
  %     \end{itemize}
  %   \end{alertblock}
  %
  %     \column{.3\textwidth}
  %
  %   \begin{center}
  %     \includegraphics[width=\textwidth]{due-tomorrow}\vspace{1em}
  %
  %     This won't work.
  %   \end{center}
  %
  % \end{columns}
  %
  % \end{frame}
  % %
  % %

  %
  %
  \subsection{Homework}
  %
  %


  \begin{frame}[t]{Course \red{mechanics}}

  \begin{columns}[T]

    \column{.6\textwidth}

    \begin{block}{Requirements}
      \begin{itemize}
        \item Read the \textbf{\emph{Stata Guide}}
        \item Practice \textbf{coding with Stata}
      \end{itemize}
    \end{block}

    \begin{alertblock}{Research project}
      \begin{itemize}
        \item Form \textbf{student groups}
        \item Code your own \textbf{data analysis}
        \item Write a \textbf{research report}
        % \item Draft, revised, final
        % \item Project management
      \end{itemize}
    \end{alertblock}

      \column{.3\textwidth}

    \begin{center}
      \includegraphics[width=\textwidth]{due-tomorrow}\vspace{1em}

      This won't work.
    \end{center}

  \end{columns}

  \end{frame}
  %
  %

  % \begin{frame}[t]{Course \red{homework}}
  %
  %   % \begin{exampleblock}{Course website}
  %   %     \url{http://f.briatte.org/teaching/quanti/}
  %   % \end{exampleblock}
  %
  %   \begin{block}{Readings}
  %     \begin{itemize}
  %       \item Urdan                 \hfill … essential textbook
  %       \item Feinstein and Thomas  \hfill … details on modelling
  %       \item Stata Guide           \hfill … practical walkthrough
  %     \end{itemize}
  %   \end{block}
  %
  %   \begin{block}{Coursework}
  %     \begin{itemize}
  %         \item \textbf{Replicate} the weekly do-files
  %         \item \textbf{Code} your own data analysis
  %         \item \textbf{Write} an empirical research paper
  %     \end{itemize}
  %   \end{block}
  %
  % \end{frame}
  % %
  % %

  %
  %
  \subsection{Logistics}
  %
  %

  \begin{frame}[t]{Course \red{logistics}}

    \begin{columns}[T]

      \column{.35\textwidth}

        \textbf{Elect a student representative!}\\[1em]

        No estimation without representation. One (wo)man, one vote.\vspace{1em}

        \textbf{Any questions so far?}\\[1em]

        Do not worry about deadlines, they will be discussed in class.

      \column{.55\textwidth}

        \includegraphics[height=.8\textheight]{vote}

    \end{columns}

  \end{frame}
  %
  %

  % =========
  % = NOTES =
  % =========

%  \section{Notes}

  % %
  % %
  % \subsection{On computers}
  %
  % \begin{frame}[t]{A note on \red{computers}}
  %
  %   \begin{columns}[T]
  %
  %     \column{.4\textwidth}
  %
  %     Despite what the general computing industry tells you, %
  %     computers are more than\\[1em]
  %
  %     \begin{itemize}
  %       \item Music players
  %       \item Facebook terminals
  %       \item Porn stashes
  %     \end{itemize}
  %
  %     \column{.4\textwidth}
  %
  %     \includegraphics[width=\textwidth]{dilbert-porn}
  %
  %   \end{columns}
  %
  %     \vspace{1em}
  %
  %     \begin{alertblock}{Important}
  %       This course requires that you \textbf{learn how to \emph{work}} with a computer. %
  %         \href{http://www.orbooks.com/catalog/program/}{Program or be programmed.}
  %     \end{alertblock}
  %
  % \end{frame}
  % %
  % %


  % %
  % %
  % \subsection{On software}
  %
  % \begin{frame}[t]{A note on \red{software}}
  %
  %   \begin{columns}[T]
  %
  %     \column{.4\textwidth}
  %
  %     We will be using \red{Stata} throughout the semester.\\[1em]
  %
  %     Software details:
  %
  %     \href{http://www.stata.com/}{stata.com}
  %
  %     \column{.5\textwidth}
  %
  %     \begin{center}
  %       \vspace{-2em}
  %       \includegraphics[width=.5\textwidth]{icon-stata12}
  %     \end{center}
  %
  %   \end{columns}
  %
  %     \vspace{1em}
  %
  %     \begin{alertblock}{Requirements}
  %
  %       To operate Stata efficiently, you need
  %
  %       \begin{itemize}
  %         \item to use a fairly recent computer with a bit of disk space
  %           % ... system requirements
  %         \item to understand how files are organized on your hard drive
  %           % ... folder paths
  %         \item to type and `run' commands that follow a specific syntax
  %           % ... keyboard shortcuts
  %       \end{itemize}
  %
  %     \end{alertblock}
  %
  % \end{frame}
  % %
  % %

  % %
  % %
  % \subsection{On slides}
  % %
  % %
  %
  % \begin{frame}[t]{A note on \red{slides}}
  %
  %   The course slides are absolutely insufficient to complete the %
  %   course requirements. You really need to do the readings.\\[1em]
  %
  %   There is no way out of it.\\[1em]
  %
  %   \includegraphics[width=\textwidth]{dilbert-ppt}
  %
  % \end{frame}
  % %
  % %

  %
  %
%  \subsection{Notes}
  %
  %

%  \begin{frame}[t]{Final notes}
%
%    \begin{alertblock}{Emails}
%      \begin{enumerate}
%        \item Start your email subject line with the ``\textbf{SRQM:}'' prefix.%
%        \item Describe the content of the email in the subject line.%
%        \item For code-related questions, \textbf{attach your Stata do-file} (do-file).%
%      \end{enumerate}
%    \end{alertblock}
%
%    \begin{block}{Google}
%      \begin{itemize}
%        \item This course requires \textbf{\href{http://mail.google.com/}{Google Mail}} %
%          and \textbf{\href{http://docs.google.com/}{Google Docs}}
%        \item This course assumes that you know how to use both tools
%        \item Your \textbf{Sciences Po account} is actually a Google account
%      \end{itemize}
%    \end{block}
%
%  \end{frame}
%  %
%  %

  % =======
  % = END =
  % =======

  \section{Welcome}

  \begin{frame}[t, plain]

    \vspace{.1\paperwidth}

    \begin{center}
      {%
      \Large \red{Welcome on board!}}\\[.1\paperwidth]

      \href{http://xkcd.com/552/}{\includegraphics[height=4cm]{xkcd-correlation}%
      }
    \end{center}

  \end{frame}
  %
  %

  % %
  % %
  % \section{Coursework}
  % %
  % %

  % \begin{frame}[t]{Coursework \red{for next week}}
  %
  %     From this week onwards, you will need Stata to have been \textbf{set up for the course}. Listen carefully in class as we go through the procedure! Then, once you are set up, open Stata.\\[1em]%
  %
  %   \begin{block}{From Stata, type the line \code{in blue}}
  %       % \comm{Get the slides.}\\
  %       % \code{srqm\_get week1.pdf}\\
  %
  %       % \comm{Get the do-file.}\\
  %       % \code{srqm\_get week1.do}\\
  %
  %       \comm{Open the do-file.}\\
  %       \code{doedit code/week1}
  %   \end{block}
  %
  %   \begin{alertblock}{Homework}
  %     \begin{itemize}
  %          \item \textbf{Read the do-file} and execute its commands.
  %          \item \textbf{Practice} searching and describing variables.
  %          \item \textbf{Start thinking} about which dataset to analyse.
  %     \end{itemize}
  %   \end{alertblock}
  %
  % \end{frame}
  % %
  % %

\end{document}

